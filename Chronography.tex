\documentclass[a4paper]{article}
\usepackage[T1,T2A]{fontenc}
\usepackage[utf8]{inputenc}
\usepackage[english,russian]{babel}
\usepackage{cmap}
\usepackage{ccicons}
\usepackage{hyperxmp}
\usepackage[colorlinks,linkcolor=blue,unicode]{hyperref}
\usepackage[left=1cm, right=1cm, top=1cm, bottom=1cm]{geometry}
\usepackage{tikz}
\usepackage{xspace}
\usepackage{graphicx}
\pdfinclusioncopyfonts=1 

\newcommand{\myhref}[2]{\textcolor{blue}{\href{#1}{\textcolor{blue}{#2}}}}

\hypersetup{%
pdftitle={%
Правители Византии XI века.По материалам "Хронографии" Михаила Пселла},
pdfauthor={Vitaly Repin},
pdfcopyright={This work is licensed under a Creative Commons Attribution-ShareAlike 3.0 Unported License},
pdfsubject={Правители Византии XI века.По материалам "Хронографии" Михаила Пселла},
pdfkeywords={Византия,Пселл,Psellos},
pdflicenseurl={http://creativecommons.org/licenses/by-sa/3.0/},
pdfcaptionwriter={Vitaly Repin},
pdfcontactcity={Espoo},
pdfcontactcountry={Finland},
pdfcontactemail={vitaly.repin@gmail.com},
pdflang={ru}
}

\pagestyle{empty}

\begin{document}

\centerline{\textsc{\bf\large Императоры и императрицы Византии XI века}}
\centerline{\sc\small по материалам <<Хронографии>> Михаила Пселла}
\vspace*{-.8cm}

\centerline{\includegraphics{graph}}

\definecolor{myblue}{HTML}{0EBFE9}
%% Adapted from: http://tex.stackexchange.com/questions/7032/good-way-to-make-textcircled-numbers
%% Can be used only inzide tikzpicture environment
\newcommand{\circled}[2]{\node[shape=circle,draw,inner sep=1pt,fill=myblue,font=\small] at (#1) {\textcolor{white}{#2}};}
%% Dtand-alone version of the command \circled
\newcommand{\circledd}[1]{\raisebox{-.1cm}{\begin{tikzpicture}\circled{0,0}{#1}\end{tikzpicture}}\xspace}

\newcounter{it}
\setcounter{it}{1}
\newcommand{\nitem}{\circledd{\arabic{it}}\addtocounter{it}{1}}

\centerline{%
\begin{tabular}{|l|l|l|}
\hline
\bf Властитель&\bf Годы царствования& \bf Годы жизни\\
\hline
\nitem Василий II Болгаробойца&11.01.976 -- 15.12.1025&958 -- 15.12.1025\\
\hline
\nitem Константин VIII&16.12.1025 -- 11.11.1028 & 960 -- 15.11.1028\\
\hline
\nitem Роман III Аргир & 12.11.1028 -- 11.04.1034 &968 -- 11.04.1034\\
\hline
\nitem Михаил IV Пафлагонский & 11.04.1034 - 10.12.1041 & 1010 -- 10.12.1041\\
\hline
\nitem Михаил V Калафат & 11.12.1041 -- 22.04.1042 & 1015 -- 24.08.1042\\
\hline
\nitem Зоя, Феодора  & 22.04.1042 -- 11.06.1042 &\\
\hline
\nitem Константин IX Мономах & 11.06.1042 -- 08.01.1055 &1000 -- 11.01.1055\\
\hline
\nitem Феодора& 08.01.1055 -- 31.08.1056 & 984 -- 31.08.1056\\
\hline
\nitem Михаил VI Стратиотик & 08.1056 -- 03.09.1057 & ? -- 1059\\
\hline
\nitem Исаак I Комнин &04.09.1057 -- 22.11.1059 & 1005 -- 31.05.1061\\
\hline
\nitem Константин X Дука & 23.11.1059 -- 23.05.1067 & 1006 -- 23.05.1067\\
\hline
\nitem Евдокия Макремволитисса & 23.05.1067 -- 01.01.1068 & 1021 -- 1096\\
\hline
\nitem Роман IV Диоген & 01.01.1068 -- 01.10.1071 & 1030 -- 1072\\
\hline
\nitem Михаил VII Дука &01.11.1071 -- 24.03.1078 & 1050 -- 1090\\
\hline
\end{tabular}}

\noindent\begin{thebibliography}{9}
\bibitem{psell} Пселл Михаил. Хронография. \textsl{М., 1978.}
\end{thebibliography}

\centerline{\small \ccbysa\ Vitaly Repin, 2014. This work is licensed under a \myhref{http://creativecommons.org/licenses/by-sa/3.0/}{Creative Commons Attribution-ShareAlike 3.0 Unported License.}}

\end{document}
